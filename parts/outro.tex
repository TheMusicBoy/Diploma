\isection{Заключение}

    В результате проведенного исследования разработана трехуровневая архитектура ORM для C++, реализующая инверсию зависимостей между объектно-ориентированной и реляционной моделями данных: ядро ORM обеспечивает типобезопасное отображение объектов на таблицы, Master абстрагирует взаимодействие с различными СУБД и оптимизирует генерацию SQL-запросов, а Migrator автоматизирует эволюцию схемы базы данных в соответствии с изменениями в объектной модели. Выбор Protocol Buffers в качестве основы для кодогенерации решает проблему отсутствия встроенной рефлексии в C++, а внедрение разных методов оптимизации обеспечивает высокую производительность, сохраняя при этом простоту использования API — что делает предложенную архитектуру оптимальным решением для высоконагруженных систем, где критичны как строгая типизация и производительность C++, так и удобство работы с реляционными данными.

\isection{Список литературы}

    [1] rezahazegh Understanding Object-Relational Impedance Mismatch — DEV, 2024.

    [2] Сагалаев И. Хранение объектов не в БД — 2005.

    [3] GeeksForGeeks What is Object-Relational Mapping (ORM) in DBMS? — 2024.

    [4] Чепайкин А. Плюсы и минусы написания запросов с ORM и на SQL — Habr, 2025.

    [5] Мини-обзор библиотек для Reflection в C++ — Habr, 2015.

    [6] Cap'N'Proto Serialization in C++ — wirepair.org, 2023.

    [7] @badcasedaily1 Два типа рефлексий в C++ — Habr, 2024.

    [8] Elton Minetto JSON vs FlatBuffers vs Protocol Buffers — DEV, 2024.

    [9] Никулин А. Apache Thrift RPC Server. Дружим C++ и Java — Habr, 2013.

    [10] Александр Фокин Обзор C++26 — cppconf.ru, 2024.

    [11] Яндекс YTsaurus — 2024.

\isection{Приложения}

\isubsection{Дневник студента}

\begin{table}[h]
\centering
\small
\begin{tabular}{|p{2.5cm}|p{3cm}|p{8cm}|p{1.5cm}|}
\hline
\textbf{Дата} & \textbf{Рабочее место} & \textbf{Краткое содержание выполняемых работ} & \textbf{Отметки руководителя} \\
\hline
27.02 - 03.03 & Удаленно & Изучение проблемы объектно-реляционного несоответствия импеданса. Сбор литературы по теме. & \\
\hline
03.03 - 08.03 & Удаленно & Изучение существующих ORM-решений для C++. Анализ их достоинств и недостатков. & \\
\hline
09.03 - 12.03 & Удаленно & Исследование возможностей рефлексии в C++. Сравнительный анализ подходов к решению проблемы отсутствия рефлексии. & \\
\hline
12.03 - 18.03 & Удаленно & Сравнение технологий кодогенерации: Protocol Buffers, FlatBuffers, Cap'n Proto. Выбор Protocol Buffers для дальнейшей работы. & \\
\hline
19.03 - 28.03 & Удаленно & Разработка концепции архитектуры ORM. Проектирование компонентов: Ядро ORM, Master, Migrator. & \\
\hline
28.03 - 10.04 & Удаленно & Начало реализации ядра ORM. Разработка базовых классов для отображения объектов на таблицы. & \\
\hline
10.04 - 22.04 & Удаленно & Разработка компонента Master для генерации SQL-запросов и работы с различными СУБД. & \\
\hline
\end{tabular}
\end{table}
