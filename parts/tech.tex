\isubsection{Выбор технологий для реализации ORM}

    \isubsubsection{Требования к современным ORM-решениям в C++}
    
        Современные ORM-решения для C++ должны эффективно преодолевать несколько ключевых проблем:
        
        \begin{itemize}
            \item \textbf{Отсутствие встроенной рефлексии} в C++, что усложняет автоматическое сопоставление классов с таблицами базы данных;
            \item \textbf{Необходимость типобезопасности} при преобразовании данных между языком программирования и базой данных;
            \item \textbf{Высокие требования к производительности} характерные для проектов на C++;
            \item \textbf{Минимизация рутинного кода} для определения схемы данных и операций с базой данных.
        \end{itemize}
    
    \isubsubsection{Подходы к решению проблемы отсутствия рефлексии}

        Рефлексия — это способность программы исследовать и модифицировать свою структуру и поведение во время выполнения. В языках со встроенной рефлексией (Java, C\#, Python) разработчики могут легко получать информацию о классах, их полях и методах, что критически важно для ORM. C++ изначально не имеет встроенных механизмов рефлексии, что требует применения альтернативных подходов.

        \textbf{Макросы и препроцессор}
        
            Макросы в C++ позволяют автоматизировать генерацию кода, необходимого для описания метаданных классов:
            
            \begin{itemize}
                \item \textbf{Принцип работы:} Определение специальных макросов, которые разворачиваются в код для регистрации полей класса и их типов;
                
                \item \textbf{Примеры использования:} 
                \begin{lstlisting}[style=cpp_style]
class User {
    DECLARE_ENTITY(User)
    DEFINE_PROPERTY(int, id)
    DEFINE_PROPERTY(std::string, name)
    DEFINE_PROPERTY(std::string, email)
};
                \end{lstlisting}
                
                \item \textbf{Преимущества:}
                \begin{itemize}
                    \item Простота реализации;
                    \item Не требует внешних инструментов;
                    \item Минимальные накладные расходы во время выполнения.
                \end{itemize}
                
                \item \textbf{Недостатки:}
                \begin{itemize}
                    \item Ухудшает читаемость кода;
                    \item Затрудняет отладку (ошибки в макросах сложно диагностировать);
                    \item Ограниченная гибкость и выразительность;
                    \item Необходимость явного определения метаданных для каждого класса.
                \end{itemize}
                
                \item \textbf{Примеры библиотек:} ODB, SOCI, SQLiteCpp используют макросы для отображения классов на таблицы БД.
            \end{itemize}

        \textbf{Шаблонное метапрограммирование}
        
            Шаблоны C++ предоставляют мощный механизм для метапрограммирования времени компиляции:
            
            \begin{itemize}
                \item \textbf{Принцип работы:} Использование шаблонов и специализаций для создания типо-зависимой логики, которая генерирует метаданные о классах на этапе компиляции;
                
                \item \textbf{Примеры использования:} 
                \begin{lstlisting}[style=cpp_style]
template <typename T>
struct Schema;

template <>
struct Schema<User> {
    static constexpr auto metadata() {
        return make_metadata(
            field("id", &User::id),
            field("name", &User::name),
            field("email", &User::email)
        );
    }
};
                \end{lstlisting}
                
                \item \textbf{Преимущества:}
                \begin{itemize}
                    \item Полная типобезопасность;
                    \item Ошибки выявляются на этапе компиляции;
                    \item Нет накладных расходов во время выполнения;
                    \item Более чистый синтаксис по сравнению с макросами.
                \end{itemize}
                
                \item \textbf{Недостатки:}
                \begin{itemize}
                    \item Увеличивает время компиляции;
                    \item Усложняет сообщения об ошибках;
                    \item Требует специализации для каждого класса;
                    \item Ограничено возможностями шаблонов C++.
                \end{itemize}
                
                \item \textbf{Примеры библиотек:} Boost.Hana, Boost.Fusion, magic\_get используют шаблонное метапрограммирование для создания подобий рефлексии.
            \end{itemize}

        \textbf{Кодогенерация на основе внешних схем}
        
            Данный подход использует внешние инструменты для генерации кода на основе схем или аннотаций:
            
            \begin{itemize}
                \item \textbf{Принцип работы:} Определение схемы данных во внешнем файле или с помощью специальных аннотаций, которые обрабатываются инструментом кодогенерации для создания необходимых классов и метаданных;
                
                \item \textbf{Примеры использования:} 
                \begin{lstlisting}[style=cpp_style]
// schema.proto
message User {
    required int32 id = 1;
    required string name = 2;
    required string email = 3;
}

// Запуск: protoc --cpp_out=. schema.proto
                \end{lstlisting}
                
                \item \textbf{Преимущества:}
                \begin{itemize}
                    \item Чистый и читаемый код без макросов;
                    \item Единое определение схемы данных;
                    \item Возможность генерации кода для разных языков;
                    \item Разделение описания данных и логики их обработки;
                    \item Возможность расширения схемы через плагины кодогенерации.
                \end{itemize}
                
                \item \textbf{Недостатки:}
                \begin{itemize}
                    \item Требует дополнительного шага в процессе сборки;
                    \item Дополнительные инструменты и зависимости;
                    \item Потенциальная сложность интеграции с IDE;
                    \item Ограничения выразительности определяемые форматом схемы.
                \end{itemize}
                
                \item \textbf{Примеры технологий:} Protocol Buffers, FlatBuffers, Cap'n Proto, Apache Thrift.
            \end{itemize}

        \textbf{Библиотеки статической рефлексии}
        
            Ряд библиотек предлагает решения для статической рефлексии в C++ без изменения стандарта языка:
            
            \begin{itemize}
                \item \textbf{Принцип работы:} Использование комбинации метапрограммирования, макросов и иногда предварительной обработки кода для создания возможностей рефлексии;
                
                \item \textbf{Примеры библиотек:}
                \begin{itemize}
                    \item \textbf{rttr (Run Time Type Reflection)} — предоставляет динамическую рефлексию для C++;
                    \item \textbf{refl-cpp} — легковесная, основанная только на заголовочных файлах библиотека для статической рефлексии;
                    \item \textbf{nameof} — позволяет получать имена типов, функций, переменных во время компиляции.
                \end{itemize}
                
                \item \textbf{Преимущества:}
                \begin{itemize}
                    \item Более полная поддержка рефлексии по сравнению с ручными методами;
                    \item Часто имеют более чистый API;
                    \item Некоторые поддерживают динамические возможности рефлексии времени выполнения.
                \end{itemize}
                
                \item \textbf{Недостатки:}
                \begin{itemize}
                    \item Дополнительные зависимости;
                    \item Возможно снижение производительности;
                    \item Не являются стандартизированными решениями.
                \end{itemize}
            \end{itemize}

        \textbf{Будущее рефлексии в C++}
        
            Стоит отметить, что рефлексия активно обсуждается в комитете по стандартизации C++:
            
            \begin{itemize}
                \item \textbf{Предложения по стандартизации:} Существуют предложения по добавлению рефлексии в стандарт C++ (например, P2996, P1306, P3096);
                
                \item \textbf{Статическая рефлексия:} Предлагаемые решения фокусируются преимущественно на статической рефлексии времени компиляции;
                
                \item \textbf{Сроки:} Возможное включение некоторых возможностей рефлексии ожидается в C++26 или более поздних версиях.
            \end{itemize}

    \isubsubsection{Почему выбор пал на кодогенерацию}

        После анализа различных подходов к преодолению отсутствия рефлексии в C++, кодогенерация представляется наиболее оптимальным решением для реализации ORM по следующим причинам:

        \begin{itemize}
            \item \textbf{Чистота кода:} В отличие от макросов, кодогенерация позволяет разделить описание схемы данных и логику приложения, что значительно повышает читаемость и поддерживаемость кода.
            
            \item \textbf{Производительность:} По сравнению с библиотеками статической рефлексии, кодогенерация обеспечивает более высокую производительность за счёт отсутствия дополнительных слоёв абстракции во время выполнения.
            
            \item \textbf{Возможность расширения:} Кодогенерация предоставляет гибкий механизм для расширения функциональности через плагины и настройки генератора, что критически важно для ORM-специфичных задач.
            
            \item \textbf{Типобезопасность:} Генерируемый код проверяется компилятором, обеспечивая полную типобезопасность на уровне языка, что превосходит возможности макросов.
            
            \item \textbf{Кросс-платформенность:} Генераторы кода обычно поддерживают несколько языков программирования, что упрощает интеграцию с другими частями системы и возможное будущее расширение функциональности.
            
            \item \textbf{Единое представление данных:} Схемы данных, определённые для кодогенерации, могут использоваться не только для ORM, но и для сериализации, RPC и документирования, создавая единую точку истины.
        \end{itemize}

        В то время как шаблонное метапрограммирование предлагает элегантные решения для некоторых аспектов проблемы, оно значительно усложняет процесс компиляции и отладки. Библиотеки статической рефлексии, хотя и предоставляют богатый API, вводят дополнительные зависимости и часто имеют ограничения в производительности.

        Кодогенерация предлагает оптимальный баланс между удобством разработки, производительностью и гибкостью, что делает её предпочтительным подходом для решения задач ORM в C++.

    \isubsubsection{Сравнение кодогенераторов для решения задач ORM}

        \textbf{Protocol Buffers}

            \begin{itemize}
                \item \textbf{Производительность}:
                \begin{itemize}
                    \item Высокоэффективная бинарная сериализация
                    \item Быстрая десериализация с минимальными затратами памяти
                    \item Компактное бинарное представление данных
                \end{itemize}

                \item \textbf{Форматы и преобразования}:
                \begin{itemize}
                    \item Встроенная поддержка преобразования в JSON и обратно
                    \item Компактный бинарный формат для эффективного хранения
                \end{itemize}

                \item \textbf{RPC и сетевое взаимодействие}:
                \begin{itemize}
                    \item Тесная интеграция с высокопроизводительным gRPC фреймворком
                    \item Поддержка двунаправленного потокового взаимодействия
                \end{itemize}

                \item \textbf{Поддержка C++}:
                \begin{itemize}
                    \item Генерация идиоматичного, эффективного C++ кода
                    \item Отличная интеграция с системами сборки (CMake, Bazel)
                \end{itemize}

                \item \textbf{Расширяемость}:
                \begin{itemize}
                    \item Гибкая система пользовательских опций для метаданных
                    \item Расширяемость через плагины к компилятору protoc
                    \item Отличная поддержка версионирования схем данных
                \end{itemize}

                \item \textbf{Экосистема}:
                \begin{itemize}
                    \item Активная поддержка со стороны Google
                    \item Обширная документация и сообщество
                    \item Проверенное временем решение с широким применением
                \end{itemize}
            \end{itemize}

        \textbf{FlatBuffers}

            \begin{itemize}
                \item \textbf{Производительность}:
                \begin{itemize}
                    \item Zero-copy доступ к данным без полной десериализации
                    \item Более быстрое чтение отдельных полей
                    \item Возможность прямого редактирования сериализованных данных
                \end{itemize}

                \item \textbf{Форматы и преобразования}:
                \begin{itemize}
                    \item Отсутствие встроенной поддержки преобразования в JSON
                    \item Больший размер сериализованных данных по сравнению с Protobuf
                \end{itemize}

                \item \textbf{RPC и сетевое взаимодействие}:
                \begin{itemize}
                    \item Базовая поддержка RPC без развитой инфраструктуры gRPC
                    \item Отсутствие встроенной поддержки потокового взаимодействия
                \end{itemize}

                \item \textbf{Поддержка C++}:
                \begin{itemize}
                    \item Хорошая поддержка C++, но менее идиоматичный API
                    \item Более сложная интеграция со сборочными системами
                \end{itemize}

                \item \textbf{Расширяемость}:
                \begin{itemize}
                    \item Ограниченные возможности расширения для ORM-специфичных метаданных
                    \item Базовая поддержка версионирования схем
                \end{itemize}

                \item \textbf{Экосистема}:
                \begin{itemize}
                    \item Меньшее сообщество и экосистема
                    \item Более ограниченная документация
                \end{itemize}
            \end{itemize}

        \textbf{Cap'n Proto}

            \begin{itemize}
                \item \textbf{Производительность}:
                \begin{itemize}
                    \item Zero-copy доступ к данным без полной десериализации
                    \item Очень быстрое чтение отдельных полей
                    \item Эффективная работа с большими структурами данных
                \end{itemize}

                \item \textbf{Форматы и преобразования}:
                \begin{itemize}
                    \item Отсутствие встроенной поддержки преобразования в JSON
                    \item Больший размер сериализованных данных по сравнению с Protobuf
                \end{itemize}

                \item \textbf{RPC и сетевое взаимодействие}:
                \begin{itemize}
                    \item Собственный RPC-фреймворк с меньшей функциональностью чем gRPC
                    \item Ограниченная поддержка потоковой передачи данных
                \end{itemize}

                \item \textbf{Поддержка C++}:
                \begin{itemize}
                    \item Хорошая поддержка C++, но менее идиоматичный API
                    \item Более сложная интеграция со сборочными системами
                \end{itemize}

                \item \textbf{Расширяемость}:
                \begin{itemize}
                    \item Ограниченные возможности расширения для ORM-специфичных метаданных
                    \item Хорошая поддержка версионирования схем
                \end{itemize}

                \item \textbf{Экосистема}:
                \begin{itemize}
                    \item Меньшее сообщество и экосистема
                    \item Более ограниченная документация
                \end{itemize}
            \end{itemize}

        \textbf{Apache Thrift}

            \begin{itemize}
                \item \textbf{Производительность}:
                \begin{itemize}
                    \item Эффективная сериализация, но медленнее чем у Protocol Buffers
                    \item Нет поддержки zero-copy доступа к данным
                    \item Средний размер сериализованных данных
                \end{itemize}

                \item \textbf{Форматы и преобразования}:
                \begin{itemize}
                    \item Поддержка нескольких форматов сериализации
                    \item Базовая поддержка преобразования в JSON
                \end{itemize}

                \item \textbf{RPC и сетевое взаимодействие}:
                \begin{itemize}
                    \item Встроенный RPC-фреймворк с меньшей производительностью чем gRPC
                    \item Ограниченная поддержка потоковой передачи данных
                \end{itemize}

                \item \textbf{Поддержка C++}:
                \begin{itemize}
                    \item Хорошая поддержка C++, но менее оптимизированный код
                    \item Более сложная интеграция со сборочными системами
                \end{itemize}

                \item \textbf{Расширяемость}:
                \begin{itemize}
                    \item Ограниченные возможности расширения для ORM-специфичных метаданных
                    \item Ограниченная поддержка версионирования схем
                \end{itemize}

                \item \textbf{Экосистема}:
                \begin{itemize}
                    \item Активное сообщество, но меньше чем у Protocol Buffers
                    \item Хорошая документация
                \end{itemize}
            \end{itemize}

        \textbf{JSON Schema}

            \begin{itemize}
                \item \textbf{Производительность}:
                \begin{itemize}
                    \item Низкая эффективность сериализации/десериализации
                    \item Отсутствие компактного представления данных
                    \item Значительно больший размер данных
                \end{itemize}

                \item \textbf{Форматы и преобразования}:
                \begin{itemize}
                    \item Нативная поддержка JSON
                    \item Отсутствие бинарного представления
                \end{itemize}

                \item \textbf{RPC и сетевое взаимодействие}:
                \begin{itemize}
                    \item Отсутствие встроенного RPC-фреймворка
                    \item Необходимость использования сторонних решений (REST, GraphQL)
                \end{itemize}

                \item \textbf{Поддержка C++}:
                \begin{itemize}
                    \item Ограниченная поддержка кодогенерации для C++
                    \item Сложная интеграция со сборочными системами
                \end{itemize}

                \item \textbf{Расширяемость}:
                \begin{itemize}
                    \item Хорошие возможности для определения метаданных
                    \item Слабая поддержка версионирования схем
                \end{itemize}

                \item \textbf{Экосистема}:
                \begin{itemize}
                    \item Широкое применение в веб-средах
                    \item Отсутствие единого стандарта кодогенерации
                \end{itemize}
            \end{itemize}

        \textbf{XML Schema (XSD)}

            \begin{itemize}
                \item \textbf{Производительность}:
                \begin{itemize}
                    \item Очень низкая эффективность сериализации/десериализации
                    \item Очень большой размер сериализованных данных
                    \item Высокие затраты CPU и памяти при обработке
                \end{itemize}

                \item \textbf{Форматы и преобразования}:
                \begin{itemize}
                    \item Нативная поддержка XML
                    \item Возможность преобразования в JSON через XSLT
                \end{itemize}

                \item \textbf{RPC и сетевое взаимодействие}:
                \begin{itemize}
                    \item Отсутствие современного встроенного RPC-фреймворка
                    \item Устаревшие протоколы (SOAP)
                \end{itemize}

                \item \textbf{Поддержка C++}:
                \begin{itemize}
                    \item Ограниченная поддержка кодогенерации для C++
                    \item Сложная интеграция со сборочными системами
                \end{itemize}

                \item \textbf{Расширяемость}:
                \begin{itemize}
                    \item Богатая система аннотаций и расширений
                    \item Избыточная сложность для большинства задач
                \end{itemize}

                \item \textbf{Экосистема}:
                \begin{itemize}
                    \item Устаревающие технологии с уменьшающимся сообществом
                    \item Применение преимущественно в старых корпоративных системах
                \end{itemize}
            \end{itemize}

        \textbf{Сравнительная таблица}

            \begin{table}[H]
            \centering
            \scriptsize
            \begin{tabular}{|l|c|c|c|c|c|c|}
            \hline
            \textbf{Критерий} & \textbf{Protocol Buffers} & \textbf{FlatBuffers} & \textbf{Cap'n Proto} & \textbf{Apache Thrift} & \textbf{JSON Schema} & \textbf{XML Schema} \\
            \hline
            Скорость сериализации & Высокая & Средняя & Высокая & Средняя & Низкая & Очень низкая \\
            \hline
            Скорость десериализации & Высокая & Очень высокая & Очень высокая & Средняя & Низкая & Очень низкая \\
            \hline
            Размер данных & Компактный & Средний & Средний & Средний & Большой & Очень большой \\
            \hline
            Поддержка JSON & Встроенная & Отсутствует & Отсутствует & Базовая & Нативная & Через конвертеры \\
            \hline
            RPC-фреймворк & gRPC & Базовый & Средний & Встроенный & Отсутствует & SOAP \\
            \hline
            Поддержка C++ & Отличная & Хорошая & Хорошая & Хорошая & Ограниченная & Ограниченная \\
            \hline
            Расширяемость для ORM & Высокая & Средняя & Средняя & Средняя & Средняя & Высокая \\
            \hline
            Зрелость и сообщество & Очень высокая & Средняя & Средняя & Высокая & Высокая & Затухающая \\
            \hline
            Простота использования & Высокая & Средняя & Средняя & Средняя & Высокая & Низкая \\
            \hline
            \end{tabular}
            \caption{Сравнение кодогенераторов для ORM в C++}
            \end{table}

    \isubsubsection{Преимущества Protocol Buffers для ORM в C++}

        \begin{itemize}
            \item \textbf{Комплексная экосистема}:
            \begin{itemize}
                \item Единая технология для описания данных, RPC и ORM
                \item Возможность использования одних и тех же моделей для базы данных, API и межсервисного взаимодействия
            \end{itemize}

            \item \textbf{Баланс производительности и удобства}:
            \begin{itemize}
                \item Высокая эффективность без избыточного усложнения API
                \item Компактное представление данных с сохранением читаемости кода
            \end{itemize}

            \item \textbf{Универсальность форматов}:
            \begin{itemize}
                \item Бинарный формат для высокопроизводительных сценариев
                \item Встроенное преобразование в JSON для интеграции с веб-технологиями
            \end{itemize}

            \item \textbf{Оптимальная поддержка для C++}:
            \begin{itemize}
                \item Идиоматичный и эффективный генерируемый код
                \item Отличная интеграция с типичными C++ системами сборки
            \end{itemize}

            \item \textbf{Расширяемость для ORM}:
            \begin{itemize}
                \item Гибкая система пользовательских опций для ORM-специфичных метаданных
                \item Возможность расширения компилятора протоколов через плагины
            \end{itemize}
        \end{itemize}

    \isubsubsection{Заключение}

        Protocol Buffers выбран как оптимальное решение для реализации ORM в C++ благодаря сочетанию высокой производительности, гибкости и развитой экосистеме. Интеграция с gRPC и поддержка JSON обеспечивают универсальность для всех аспектов приложения. По сравнению с другими решениями, Protocol Buffers предлагает для меня наилучший баланс качеств.
