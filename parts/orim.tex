\isubsection{Объектно-реляционное несоответствие импеданса}

    Объектно-реляционное несоответствие импеданса -- это набор трудностей, которые возникают при передаче данных между реляционными базами данных и моделями объектно-ориентированного программирования.

    \isubsubsection{Различия концепций моделей}

        Объектно-ориентированный подход, с точки зрения математики, представляют собой направленные графы, где объект ссылается на другие дочерние. Реляционные модели основаны на кортежах в отношениях (таблицах) с использованием реляционной алгебры. Кортеж — это набор типизированных полей данных (строк). В реляционной модели связи обратимы, представляя собой ненаправленные графы.

    \isubsubsection{Различия в объектно-ориентированных концепциях}

        \textbf{Инкапсуляция:} В ООП инкапсуляция скрывает внутреннее состояние объектов. Свойства объекта доступны только через реализованные интерфейсы.

        \textbf{Доступность:} «Приватное» и «публичное» в реляционных системах зависит от потребностей. В ООП это абсолютно зависит от классов. Эта относительность и абсолютизм классификаций и характеристик противоречат друг другу.

        \textbf{Интерфейсы, классы, наследование и полиморфизм:} В ООП модели объекты должны реализовывать интерфейсы для предоставления доступа к своему внутреннему состоянию. В реляционной модели используются представления (views) для изменения перспективы и ограничений. Реляционная модель не имеет таких ООП концепций, как классы, наследование и полиморфизм.

    \isubsubsection{Различия в типах данных}

        Реляционная модель запрещает ссылки, в то время как ООП модель активно их использует. Скалярные типы также часто различаются между моделями, что затрудняет отображение.

        SQL поддерживает строки с максимальной длиной, обеспечивая большую производительность, и правила сортировки. В ООП библиотеках правила сортировки реализуются только через отдельные процедуры сортировки, а строки ограничены только объемом доступной памяти.

    \isubsubsection{Манипулятивные различия}

        Реляционная модель использует стандартизированные операторы для работы с данными, в то время как ООП использует императивный подход для каждого класса и каждого случая. Любая декларативная поддержка ООП предназначена для списков и хеш-таблиц, в отличие от множеств в реляционной модели.

    \isubsubsection{Транзакционные различия}

        В реляционной модели единицей работы является транзакция, которая превышает по размеру любые методы ООП классов. Транзакции могут включать произвольные комбинации операций с данными, тогда как в ООП модели есть только отдельные присваивания примитивным полям. ООП модель не обеспечивает изоляцию и долговечность, так что атомарность и согласованность доступны только на уровне примитивов.

    \isubsubsection{Различия в философии}

        \begin{itemize}
            \item \textbf{Декларативный и императивный подходы} — Реляционная модель использует декларативный подход к данным, в то время как ООП использует императивное поведение.
            
            \item \textbf{Привязка к схеме} — Реляционная модель ограничивает строки их схемами сущностей. ООП наследование (древовидное или нет) аналогично.
            
            \item \textbf{Правила доступа} — в реляционных системах используются стандартизированные операторы, в то время как в ОО-классах есть отдельные методы. ОО-системы более выразительны, но реляционные системы отличаются математической логикой, целостностью и согласованностью дизайна.
            
            \item \textbf{Идентичность объекта} — Два изменяемых объекта с одинаковым состоянием различны. Реляционная модель игнорирует эту уникальность и должна искусственно создавать её с помощью кандидатных ключей, что является плохой практикой, если этот идентификатор не существует в реальном мире. Идентичность постоянна в реляционной модели, но может быть временной в ООП.
            
            \item \textbf{Нормализация} — ООП пренебрегает реляционной нормализацией.
            
            \item \textbf{Наследование схемы} — Реляционные схемы отвергают иерархическое наследование ООП. Реляционная модель принимает более мощную теорию множеств. Существует непопулярное недревовидное (не-Java) ООП, но оно сложнее, чем реляционная алгебра.
            
            \item \textbf{Структура и поведение} — ООП фокусируется на структуре (поддерживаемость, расширяемость, повторное использование, безопасность). Реляционная модель фокусируется на поведении (эффективность, адаптивность, отказоустойчивость, живучесть, логическая целостность и т.д.).
        \end{itemize}

    \isubsubsection{Итог}

        Использование реляционной базы данных для хранения объектно-ориентированных данных приводит к семантическому разрыву, который заставляет программистов постоянно преобразовывать данные между двумя разными формами представления. Эта необходимость не только снижает производительность разработки, но и создает дополнительные трудности, поскольку обе формы данных накладывают ограничения друг на друга.

        Поскольку в реляционных базах данных информация хранится в виде набора таблиц с простыми данными, часто для хранения одного объекта необходимо использовать несколько таблиц, что требует применения операции JOIN для получения всей информации об объекте. Например, для хранения данных адресной книги могут использоваться как минимум две таблицы: люди и адреса, а возможно, и дополнительная таблица с телефонными номерами.

        Традиционные СУБД обычно не оптимизированы для эффективной работы с объектно-ориентированными структурами через множество последовательных запросов. Последовательные запросы могут быть затратными и приводить к избыточным операциям чтения и записи. Один сложный запрос, охватывающий все связанные данные, обычно выполняется быстрее серии простых запросов.
